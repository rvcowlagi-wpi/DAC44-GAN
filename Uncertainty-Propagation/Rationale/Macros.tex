%=============================================================================================
%----- Commonly used math symbols
\newcommand{\real}[1][]{\mathbb{R}^{#1}}                                % Set of reals
\newcommand{\nat}[1][]{\mathbb{N}^{#1}}                                 % Set of natural #s
\newcommand{\integer}[1][]{\mathbb{Z}^{#1}}                             % Set of integers
\newcommand{\circspace}[1][]{\mathbb{S}^{#1}}                           % S^n space
\newcommand{\leb}[1][]{\mathbb{L}^{#1}}

\newcommand{\defeq}{:=}                                                 % Definition 
%(assignment)
\newcommand{\msub}[1]{_\mathrm{#1}}                                     % Roman subscript
\newcommand{\msup}[1]{^\mathrm{#1}}                                     % Roman superscript

\newcommand{\trans}[1]{#1^{\intercal}}                                 % Transpose
\newcommand{\eye}[1]{I_{(#1)}}                                          % Identity matrix
\newcommand{\zero}[2]{{\mathbf 0}_{(#1)\times(#2)}}                     % Zero matrix
\newcommand{\inv}[1]{{#1}^{-1}}                                         % Inverse (matrix)
\newcommand{\rinv}[1]{#1^{-1}}                                          % Inverse (scalar)
\newcommand{\abs}[1]{\left| #1 \right|}                                 % Absolute value
\newcommand{\linspan}[1]{\mathsf{span} \left\{#1\right\}}               % Span
\newcommand{\myfracA}[2]{\displaystyle{\frac{#1}{#2}}}                  % Fraction, 
%display-style
\newcommand{\myfracB}[2]{{#1}/{#2}}                                     % Fraction, inline

\newcommand{\mydiff}[2]{\frac{\mathrm{d}{#1}}{\mathrm{d}{#2}}}      	% Derivative, plain
\newcommand{\mydiffp}[2]{{#1}'}      	% Derivative, prime
\newcommand{\mydiffn}[3]{{#1}^{(#3)}}      	% Derivative, prime
\newcommand{\mydiffnA}[3]{\frac{\mathrm{d}^{#3}{#1}}{\mathrm{d}{#2}^{#3}}}      	% 
%Derivative, plain
\newcommand{\mydiffA}[2]{\myfracA{\mathrm{d}}{\mathrm{d}{#2}}{#1}}      % Derivative, 
%display-sty
\newcommand{\mydiffB}[2]{\mathrm{d}{#1}/\mathrm{d}{#2}}                 % Derivative, inline
\newcommand{\df}{\mathrm{d}}                                            % Differential
\newcommand{\parder}[2]{\frac{\partial #1}{\partial #2}}                % Partial derv.
\newcommand{\pardern}[3]{\frac{\partial^{#3} #1}{\partial {#2}^{#3}}}   % Partial derv.
\newcommand{\parders}[2]{{#1}_{#2}}   % Partial derv.
\newcommand{\innerp}[2]{\left\langle {#1}, {#2} \right\rangle}     		% Inner product
\newcommand{\dotprod}[2]{ {#1} \cdot {#2} }
\newcommand{\hessian}[1]{\mathbf{H}(#1)}

\newcommand{\scos}[1]{C_{#1}}
\newcommand{\ssin}[1]{S_{#1}}
\newcommand{\stan}[1]{T_{#1}}

\newcommand{\acos}[1]{\cos^{-1}\left(#1\right)}                         % arccos
\newcommand{\asin}[1]{\sin^{-1}\left(#1\right)}                         % arcsin
\newcommand{\atan}[1]{\tan^{-1}\left(#1\right)}                         % arctan
\newcommand{\ppbt}{\frac{\pi}{2}}                                       % +pi/2
\newcommand{\mpbt}{-\frac{\pi}{2}}                                      % -pi/2
\newcommand{\half}{ {\textstyle{\frac{1}{2}}} }							% Half
\newcommand{\quarter}{ {\textstyle{\frac{1}{4}}} }							% Half

\newcommand{\clint}[2]{\left[#1, #2\right]}                             % Closed interval
\newcommand{\opint}[2]{\left(#1, #2\right)}                             % Open interval
\newcommand{\lopint}[2]{(#1, #2]}                                       % Left-open interval
\newcommand{\ropint}[2]{\left[#1, #2\right)}                            % Right-open interval

\renewcommand{\leq}{\leqslant}                                          % Slanted leq
\renewcommand{\geq}{\geqslant}                                          % Slanted geq

\newcommand{\myinfty}{\chi}

\newcommand{\tento}[1]{\times 10^{#1}}
\newcommand{\msq}{m/s\textsuperscript{2}}

%=============================================================================================
%----- Shortcuts for commonly used phrases and mark-ups
\newcommand{\mydef}[1]{{\textbf{#1}}\index{#1}}                          
\newcommand{\markup}[1]{ {\hl{ #1 }} }
\newcommand{\rcmargin}[2]{{\color{blue}{#1}}
	\marginpar{\scriptsize\color{blue} #2}}
\newcommand{\rccomment}[2]{{\textcolor{red}{\textul{#1}}} \markup{#2}}

\newcommand{\rhs}{\textsc{r.h.s.}}                                      % Right hand side
\newcommand{\lhs}{\textsc{l.h.s.}}                                      % Left hand side
\newcommand{\wrt}{with respect to}                                      % or w.r.t.
\newcommand{\comingsoon}{\includegraphics[width=0.2\columnwidth]{\figpath/coming-soon}}

%=============================================================================================
%----- Acronyms
\newcommand{\ppl}{ath-planning}
\newcommand{\mpl}{otion-planning}
\newcommand{\tpl}{ask-planning}
\newcommand{\rpl}{oute-planning}
\newcommand{\cipl}{ondition-informed m\mpl}
\newcommand{\ips}{nteractive planning and sensing}
\newcommand{\mr}{ultiresolution}                                      	% Multi-resolution
\newcommand{\slam}{{SLAM}}                                       		% SLAM
\newcommand{\rrt}{{RRT}}
\newcommand{\rrts}{{RRT}$^*$}
\newcommand{\syclop}{\textsc{s}y\textsc{cl}o\textsc{p}}
\newcommand{\cbta}{CBTA}
\newcommand{\nh}{onholonomic}                                           % nonholonomic
\newcommand{\dwt}{DWT}
\newcommand{\astar}{A$^*$}
\newcommand{\cpsf}{yber-physical system}
\newcommand{\vvc}{VV\&C}
\newcommand{\matlab}{MATLAB\textsuperscript{\textregistered}}
\newcommand{\phyphox}{Phyphox\textsuperscript{\textregistered}}
\def\cps{cyberphysical}
\def\Cps{Cyberphysical}

\def\canvas{\textit{Canvas}}
\def\teams{\textit{Teams}}

%=============================================================================================
%----- Algorithm
\newcommand{\algtoprule}{\hrule depth 2pt width \columnwidth \vspace{1ex}}
\newcommand{\alghead}[1]{\begin{center} \textbf{{#1}} \end{center} \vspace{-\baselineskip}}
\newcommand{\algheadnoio}[1]{\begin{center} \textbf{{#1}} \end{center} \vspace{-1ex}}
\newcommand{\algmidrule}{\hrule depth 0.5pt width \columnwidth \vspace{1ex}}
\newcommand{\algbottomrule}{\vspace{1ex} \hrule depth 2pt width \columnwidth}
\newcommand{\algio}[2]{\begin{center} \noindent \textbf{Input}: #1 \qquad \textbf{Output:} #2 
	\end{center} \vspace{-0.25\baselineskip} }
\newcommand{\algproc}[1]{\raggedright \textbf{procedure}~\textsc{#1}}%


%=============================================================================================
%----- List formatting
\newcommand{\listformat}{\vspace{-0.35\baselineskip}\itemsep 0pt}
\newcommand{\tightlistformat}{\vspace{-0.35\baselineskip}\itemsep -0.5ex}
\newcommand{\looselistformat}{\vspace{-0.4\baselineskip}\itemsep 1ex}



%=============================================================================================
%----- Figure and equation references

\newcommand{\eqnnt}[1]{\hyperref[#1]{(\ref*{#1})}}
\newcommand{\eqnsnt}[2]{\hyperref[#1]{(\ref*{#1})}
	and~\hyperref[#2]{(\ref*{#2})}}
\newcommand{\eqnsernt}[2]{\hyperref[#1]{(\ref*{#1})}--\hyperref[#2]{(\ref*{#2})}}


\newcommand{\eqn}[1]{\hyperref[#1]{Eqn.~(\ref*{#1})}}
\newcommand{\eqns}[2]{\hyperref[#1]{Eqns.~(\ref*{#1})} and~\hyperref[#2]{(\ref*{#2})}}
\newcommand{\eqnser}[2]{\hyperref[#1]{Eqns.~(\ref*{#1})}--\hyperref[#2]{(\ref*{#2})}}
\newcommand{\eqnf}[1]{\hyperref[#1]{Equation~(\ref*{#1})}}
\newcommand{\eqnfs}[2]{\hyperref[#1]{Equations~(\ref*{#1})} and~\hyperref[#2]{(\ref*{#2})}}

\newcommand{\scn}[1]{\hyperref[#1]{Section~\ref*{#1}}}
\newcommand{\scns}[2]{\hyperref[#1]{Sections~\ref*{#1}} and~\hyperref[#2]{\ref*{#2}}}
\newcommand{\scnser}[2]{\hyperref[#1]{Sections~\ref*{#1}}--\hyperref[#2]{\ref*{#2}}}

\newcommand{\fig}[1]{\hyperref[#1]{Fig.~\ref*{#1}}}
\newcommand{\figs}[2]{\hyperref[#1]{Figs.~\ref*{#1}} and~\hyperref[#2]{\ref*{#2}}}
\newcommand{\figser}[2]{\hyperref[#1]{Figs.~\ref*{#1}}--\hyperref[#2]{\ref*{#2}}}
\newcommand{\figf}[1]{\hyperref[#1]{Figure~\ref*{#1}}}
\newcommand{\figfs}[2]{\hyperref[#1]{Figures~\ref*{#1}} and~\hyperref[#2]{\ref*{#2}}}
\newcommand{\figfser}[2]{\hyperref[#1]{Figures~\ref*{#1}}--\hyperref[#2]{\ref*{#2}}}

\newcommand{\tbl}[1]{\hyperref[#1]{Table~\ref*{#1}}}
\newcommand{\tbls}[2]{\hyperref[#1]{Tables~\ref*{#1}} and~\hyperref[#2]{\ref*{#2}}}

\newcommand{\apx}[1]{\hyperref[#1]{Appendix~\ref*{#1}}}

\newcommand{\exmpl}[1]{\hyperref[#1]{Example~\ref*{#1}}}
\newcommand{\exmpls}[2]{\hyperref[#1]{Examples~\ref*{#1}} and~\hyperref[#2]{\ref*{#2}}}
\newcommand{\exmplser}[2]{\hyperref[#1]{Examples~\ref*{#1}}--\hyperref[#2]{(\ref*{#2})}}

\newcommand{\chp}[1]{\hyperref[#1]{Chapter~\ref*{#1}}}
\newcommand{\chps}[2]{\hyperref[#1]{Chapters~\ref*{#1}} and~\hyperref[#2]{(\ref*{#2})}}
\newcommand{\chpser}[2]{\hyperref[#1]{Chapters~\ref*{#1}}--\hyperref[#2]{(\ref*{#2})}}

\newcommand{\prb}[1]{\hyperref[#1]{Problem~\ref*{#1}}}
\newcommand{\prp}[1]{\hyperref[#1]{Prop.~\ref*{#1}}}
\newcommand{\prpf}[1]{\hyperref[#1]{Proposition~\ref*{#1}}}

\newcommand{\algref}[1]{\hyperref[#1]{Algorithm~\ref*{#1}}}

\newcommand{\thmref}[1]{\hyperref[#1]{Theorem~\ref*{#1}}}
\newcommand{\thmsref}[2]{\hyperref[#1]{Theorems~\ref*{#1}} and~\hyperref[#2]{\ref*{#2}}}
\newcommand{\thmserref}[2]{\hyperref[#1]{Theorems~\ref*{#1}}--\hyperref[#2]{\ref*{#2}}}


\newcommand{\algline}[1]{\hyperref[#1]{Line~\ref*{#1}}}
\newcommand{\alglines}[2]{\hyperref[#1]{Lines~\ref*{#1}} and~\hyperref[#2]{\ref*{#2}}}
\newcommand{\alglineser}[2]{\hyperref[#1]{Lines~\ref*{#1}}--\hyperref[#2]{\ref*{#2}}}
\newcommand{\algassign}{\defeq}


\renewcommand{\vec}[1]{\boldsymbol{#1}}


\definecolor{wpi-gray}{RGB}{169,176,183}
\newcommand{\makegray}[1]{\textcolor{wpi-gray}{#1}}
