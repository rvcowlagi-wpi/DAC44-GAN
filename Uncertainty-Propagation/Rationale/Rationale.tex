\documentclass[aspectratio=169,handout]{beamer} 	% <== Use this during development to save time
%\documentclass[aspectratio=169]{beamer}		% <== Use this for final compilation (comment out previous)


\usepackage{color}
\usepackage{subfigure}
\usepackage{soul}
\usepackage{upgreek}
\usepackage{tikz}
\usepackage{ctable}


\mode<presentation>

%========== Presentation Theme ===========
\definecolor{wpi-red}{RGB}{175, 43, 55}
\definecolor{ace-blue}{RGB}{31,52,77}
\definecolor{ace-yellow}{RGB}{254,250,201}
\definecolor{my-gray}{gray}{0.9}

\usetheme{boxes}
\usecolortheme{orchid}

\setbeamercolor{title}{fg=ace-yellow}
\setbeamercolor{frametitle}{fg=white,bg=ace-blue}
\setbeamercolor{alerted text}{fg=wpi-red}
\setbeamercolor{normal text}{fg=black}
\setbeamercolor{item}{fg=black}
\setbeamercolor{footer}{fg=white,bg=ace-blue}
\setbeamercolor{author in head/foot}{fg=white,bg=ace-blue}
\setbeamercolor{title in head/foot}{fg=ace-blue,bg=my-gray}

\setbeamerfont{frametitle}{size=\Large}
\setbeamerfont{alerted text}{shape=\itshape}

\usepackage{charter}
\usepackage[T1]{fontenc}
\usefonttheme{serif}
%\usefonttheme[onlymath]{serif}


\setbeamertemplate{frametitle}[default][center]
\setbeamertemplate{navigation symbols}{}
\setbeamertemplate{caption}{\centering\tiny\insertcaption\par}


\setlength{\fboxrule}{1.5pt}
\setlength\abovecaptionskip{-0.3\baselineskip}
\setlength\belowcaptionskip{0.25\baselineskip}

\addtobeamertemplate{frametitle}{\vspace*{0\baselineskip}}{\vspace*{-0.06\textwidth}
	\hspace*{0.9345\textwidth}\includegraphics[width=0.1\textwidth]{Figures/ace-logo}\vspace*{-0.06\textwidth}}


\setbeamercolor{block title example}{fg=white,	bg=ace-blue}
\setbeamercolor{block title}{fg=white,	bg=wpi-red}
\setbeamercolor{block body example}{parent=normal text,use=block title,bg=block title.bg!10!bg}
%\setbeamercolor{date in head/foot}{fg=wpi-red}


\setbeamersize{text margin left=3ex,text margin right=3ex}

\newcommand{\mathalert}[1]{{\color{wpi-red}#1}}
\newcommand{\figcaption}[1]{\parbox{\columnwidth}{\centering \tiny\linespread{0.5} #1}{}}

\setlength{\fboxrule}{1pt}
\setlength{\fboxsep}{0pt}



\setbeamertemplate{footline}{
	\leavevmode%
	\hbox{%
		\begin{beamercolorbox}[wd=.2\paperwidth,ht=2.25ex,dp=1ex,left]{author in head/foot}%
			\usebeamerfont{author in head/foot} \hspace*{3ex}\insertshortauthor
		\end{beamercolorbox}%
		\begin{beamercolorbox}[wd=.6\paperwidth,ht=2.25ex,dp=1ex,center]{title in head/foot}%
			\usebeamerfont{title in head/foot} \insertshorttitle
		\end{beamercolorbox}%
		\begin{beamercolorbox}[wd=0.2\paperwidth,ht=2.25ex,dp=1ex,right]{footer}%
			%			\usebeamerfont{wpi-red}\insertshortdate{}\hspace*{2em}
			\insertframenumber{} / \inserttotalframenumber\hspace*{3ex} 
	\end{beamercolorbox}}%
	\vskip0pt%
}
%=============================================================================================
%----- Commonly used math symbols
\newcommand{\real}[1][]{\mathbb{R}^{#1}}                                % Set of reals
\newcommand{\nat}[1][]{\mathbb{N}^{#1}}                                 % Set of natural #s
\newcommand{\integer}[1][]{\mathbb{Z}^{#1}}                             % Set of integers
\newcommand{\circspace}[1][]{\mathbb{S}^{#1}}                           % S^n space
\newcommand{\leb}[1][]{\mathbb{L}^{#1}}

\newcommand{\defeq}{:=}                                                 % Definition 
%(assignment)
\newcommand{\msub}[1]{_\mathrm{#1}}                                     % Roman subscript
\newcommand{\msup}[1]{^\mathrm{#1}}                                     % Roman superscript

\newcommand{\trans}[1]{#1^{\intercal}}                                 % Transpose
\newcommand{\eye}[1]{I_{(#1)}}                                          % Identity matrix
\newcommand{\zero}[2]{{\mathbf 0}_{(#1)\times(#2)}}                     % Zero matrix
\newcommand{\inv}[1]{{#1}^{-1}}                                         % Inverse (matrix)
\newcommand{\rinv}[1]{#1^{-1}}                                          % Inverse (scalar)
\newcommand{\abs}[1]{\left| #1 \right|}                                 % Absolute value
\newcommand{\linspan}[1]{\mathsf{span} \left\{#1\right\}}               % Span
\newcommand{\myfracA}[2]{\displaystyle{\frac{#1}{#2}}}                  % Fraction, 
%display-style
\newcommand{\myfracB}[2]{{#1}/{#2}}                                     % Fraction, inline

\newcommand{\mydiff}[2]{\frac{\mathrm{d}{#1}}{\mathrm{d}{#2}}}      	% Derivative, plain
\newcommand{\mydiffp}[2]{{#1}'}      	% Derivative, prime
\newcommand{\mydiffn}[3]{{#1}^{(#3)}}      	% Derivative, prime
\newcommand{\mydiffnA}[3]{\frac{\mathrm{d}^{#3}{#1}}{\mathrm{d}{#2}^{#3}}}      	% 
%Derivative, plain
\newcommand{\mydiffA}[2]{\myfracA{\mathrm{d}}{\mathrm{d}{#2}}{#1}}      % Derivative, 
%display-sty
\newcommand{\mydiffB}[2]{\mathrm{d}{#1}/\mathrm{d}{#2}}                 % Derivative, inline
\newcommand{\df}{\mathrm{d}}                                            % Differential
\newcommand{\parder}[2]{\frac{\partial #1}{\partial #2}}                % Partial derv.
\newcommand{\pardern}[3]{\frac{\partial^{#3} #1}{\partial {#2}^{#3}}}   % Partial derv.
\newcommand{\parders}[2]{{#1}_{#2}}   % Partial derv.
\newcommand{\innerp}[2]{\left\langle {#1}, {#2} \right\rangle}     		% Inner product
\newcommand{\dotprod}[2]{ {#1} \cdot {#2} }
\newcommand{\hessian}[1]{\mathbf{H}(#1)}

\newcommand{\scos}[1]{C_{#1}}
\newcommand{\ssin}[1]{S_{#1}}
\newcommand{\stan}[1]{T_{#1}}

\newcommand{\acos}[1]{\cos^{-1}\left(#1\right)}                         % arccos
\newcommand{\asin}[1]{\sin^{-1}\left(#1\right)}                         % arcsin
\newcommand{\atan}[1]{\tan^{-1}\left(#1\right)}                         % arctan
\newcommand{\ppbt}{\frac{\pi}{2}}                                       % +pi/2
\newcommand{\mpbt}{-\frac{\pi}{2}}                                      % -pi/2
\newcommand{\half}{ {\textstyle{\frac{1}{2}}} }							% Half
\newcommand{\quarter}{ {\textstyle{\frac{1}{4}}} }							% Half

\newcommand{\clint}[2]{\left[#1, #2\right]}                             % Closed interval
\newcommand{\opint}[2]{\left(#1, #2\right)}                             % Open interval
\newcommand{\lopint}[2]{(#1, #2]}                                       % Left-open interval
\newcommand{\ropint}[2]{\left[#1, #2\right)}                            % Right-open interval

\renewcommand{\leq}{\leqslant}                                          % Slanted leq
\renewcommand{\geq}{\geqslant}                                          % Slanted geq

\newcommand{\myinfty}{\chi}

\newcommand{\tento}[1]{\times 10^{#1}}
\newcommand{\msq}{m/s\textsuperscript{2}}

%=============================================================================================
%----- Shortcuts for commonly used phrases and mark-ups
\newcommand{\mydef}[1]{{\textbf{#1}}\index{#1}}                          
\newcommand{\markup}[1]{ {\hl{ #1 }} }
\newcommand{\rcmargin}[2]{{\color{blue}{#1}}
	\marginpar{\scriptsize\color{blue} #2}}
\newcommand{\rccomment}[2]{{\textcolor{red}{\textul{#1}}} \markup{#2}}

\newcommand{\rhs}{\textsc{r.h.s.}}                                      % Right hand side
\newcommand{\lhs}{\textsc{l.h.s.}}                                      % Left hand side
\newcommand{\wrt}{with respect to}                                      % or w.r.t.
\newcommand{\comingsoon}{\includegraphics[width=0.2\columnwidth]{\figpath/coming-soon}}

%=============================================================================================
%----- Acronyms
\newcommand{\ppl}{ath-planning}
\newcommand{\mpl}{otion-planning}
\newcommand{\tpl}{ask-planning}
\newcommand{\rpl}{oute-planning}
\newcommand{\cipl}{ondition-informed m\mpl}
\newcommand{\ips}{nteractive planning and sensing}
\newcommand{\mr}{ultiresolution}                                      	% Multi-resolution
\newcommand{\slam}{{SLAM}}                                       		% SLAM
\newcommand{\rrt}{{RRT}}
\newcommand{\rrts}{{RRT}$^*$}
\newcommand{\syclop}{\textsc{s}y\textsc{cl}o\textsc{p}}
\newcommand{\cbta}{CBTA}
\newcommand{\nh}{onholonomic}                                           % nonholonomic
\newcommand{\dwt}{DWT}
\newcommand{\astar}{A$^*$}
\newcommand{\cpsf}{yber-physical system}
\newcommand{\vvc}{VV\&C}
\newcommand{\matlab}{MATLAB\textsuperscript{\textregistered}}
\newcommand{\phyphox}{Phyphox\textsuperscript{\textregistered}}
\def\cps{cyberphysical}
\def\Cps{Cyberphysical}

\def\canvas{\textit{Canvas}}
\def\teams{\textit{Teams}}

%=============================================================================================
%----- Algorithm
\newcommand{\algtoprule}{\hrule depth 2pt width \columnwidth \vspace{1ex}}
\newcommand{\alghead}[1]{\begin{center} \textbf{{#1}} \end{center} \vspace{-\baselineskip}}
\newcommand{\algheadnoio}[1]{\begin{center} \textbf{{#1}} \end{center} \vspace{-1ex}}
\newcommand{\algmidrule}{\hrule depth 0.5pt width \columnwidth \vspace{1ex}}
\newcommand{\algbottomrule}{\vspace{1ex} \hrule depth 2pt width \columnwidth}
\newcommand{\algio}[2]{\begin{center} \noindent \textbf{Input}: #1 \qquad \textbf{Output:} #2 
	\end{center} \vspace{-0.25\baselineskip} }
\newcommand{\algproc}[1]{\raggedright \textbf{procedure}~\textsc{#1}}%


%=============================================================================================
%----- List formatting
\newcommand{\listformat}{\vspace{-0.35\baselineskip}\itemsep 0pt}
\newcommand{\tightlistformat}{\vspace{-0.35\baselineskip}\itemsep -0.5ex}
\newcommand{\looselistformat}{\vspace{-0.4\baselineskip}\itemsep 1ex}



%=============================================================================================
%----- Figure and equation references

\newcommand{\eqnnt}[1]{\hyperref[#1]{(\ref*{#1})}}
\newcommand{\eqnsnt}[2]{\hyperref[#1]{(\ref*{#1})}
	and~\hyperref[#2]{(\ref*{#2})}}
\newcommand{\eqnsernt}[2]{\hyperref[#1]{(\ref*{#1})}--\hyperref[#2]{(\ref*{#2})}}


\newcommand{\eqn}[1]{\hyperref[#1]{Eqn.~(\ref*{#1})}}
\newcommand{\eqns}[2]{\hyperref[#1]{Eqns.~(\ref*{#1})} and~\hyperref[#2]{(\ref*{#2})}}
\newcommand{\eqnser}[2]{\hyperref[#1]{Eqns.~(\ref*{#1})}--\hyperref[#2]{(\ref*{#2})}}
\newcommand{\eqnf}[1]{\hyperref[#1]{Equation~(\ref*{#1})}}
\newcommand{\eqnfs}[2]{\hyperref[#1]{Equations~(\ref*{#1})} and~\hyperref[#2]{(\ref*{#2})}}

\newcommand{\scn}[1]{\hyperref[#1]{Section~\ref*{#1}}}
\newcommand{\scns}[2]{\hyperref[#1]{Sections~\ref*{#1}} and~\hyperref[#2]{\ref*{#2}}}
\newcommand{\scnser}[2]{\hyperref[#1]{Sections~\ref*{#1}}--\hyperref[#2]{\ref*{#2}}}

\newcommand{\fig}[1]{\hyperref[#1]{Fig.~\ref*{#1}}}
\newcommand{\figs}[2]{\hyperref[#1]{Figs.~\ref*{#1}} and~\hyperref[#2]{\ref*{#2}}}
\newcommand{\figser}[2]{\hyperref[#1]{Figs.~\ref*{#1}}--\hyperref[#2]{\ref*{#2}}}
\newcommand{\figf}[1]{\hyperref[#1]{Figure~\ref*{#1}}}
\newcommand{\figfs}[2]{\hyperref[#1]{Figures~\ref*{#1}} and~\hyperref[#2]{\ref*{#2}}}
\newcommand{\figfser}[2]{\hyperref[#1]{Figures~\ref*{#1}}--\hyperref[#2]{\ref*{#2}}}

\newcommand{\tbl}[1]{\hyperref[#1]{Table~\ref*{#1}}}
\newcommand{\tbls}[2]{\hyperref[#1]{Tables~\ref*{#1}} and~\hyperref[#2]{\ref*{#2}}}

\newcommand{\apx}[1]{\hyperref[#1]{Appendix~\ref*{#1}}}

\newcommand{\exmpl}[1]{\hyperref[#1]{Example~\ref*{#1}}}
\newcommand{\exmpls}[2]{\hyperref[#1]{Examples~\ref*{#1}} and~\hyperref[#2]{\ref*{#2}}}
\newcommand{\exmplser}[2]{\hyperref[#1]{Examples~\ref*{#1}}--\hyperref[#2]{(\ref*{#2})}}

\newcommand{\chp}[1]{\hyperref[#1]{Chapter~\ref*{#1}}}
\newcommand{\chps}[2]{\hyperref[#1]{Chapters~\ref*{#1}} and~\hyperref[#2]{(\ref*{#2})}}
\newcommand{\chpser}[2]{\hyperref[#1]{Chapters~\ref*{#1}}--\hyperref[#2]{(\ref*{#2})}}

\newcommand{\prb}[1]{\hyperref[#1]{Problem~\ref*{#1}}}
\newcommand{\prp}[1]{\hyperref[#1]{Prop.~\ref*{#1}}}
\newcommand{\prpf}[1]{\hyperref[#1]{Proposition~\ref*{#1}}}

\newcommand{\algref}[1]{\hyperref[#1]{Algorithm~\ref*{#1}}}

\newcommand{\thmref}[1]{\hyperref[#1]{Theorem~\ref*{#1}}}
\newcommand{\thmsref}[2]{\hyperref[#1]{Theorems~\ref*{#1}} and~\hyperref[#2]{\ref*{#2}}}
\newcommand{\thmserref}[2]{\hyperref[#1]{Theorems~\ref*{#1}}--\hyperref[#2]{\ref*{#2}}}


\newcommand{\algline}[1]{\hyperref[#1]{Line~\ref*{#1}}}
\newcommand{\alglines}[2]{\hyperref[#1]{Lines~\ref*{#1}} and~\hyperref[#2]{\ref*{#2}}}
\newcommand{\alglineser}[2]{\hyperref[#1]{Lines~\ref*{#1}}--\hyperref[#2]{\ref*{#2}}}
\newcommand{\algassign}{\defeq}


\renewcommand{\vec}[1]{\boldsymbol{#1}}


\definecolor{wpi-gray}{RGB}{169,176,183}
\newcommand{\makegray}[1]{\textcolor{wpi-gray}{#1}}





\begin{document}



\title[] 					% <= Short title here for footer on each slide
{\LARGE \bf Thoughts about Generative Models in Control Systems \\   	% <= Presentation title
	{\small \it }} 	% <= Optional subtitle
\author[Prof. R. V. Cowlagi]{{Prof. Raghvendra V. Cowlagi}\inst{}} 		% <== Author information
\institute[]{\inst{}Aerospace Engineering Department,\\
	Worcester Polytechnic Institute, Worcester, MA.\\
	\textbf{rvcowlagi@wpi.edu} \qquad \qquad		% <== Contact information
	\textbf{wpi.edu/$\sim$rvcowlagi} \qquad  		% <== Contact information
	\qquad \textbf{HL 247}}


%%%%%%%%%%%%%%%% TITLE PAGE %%%%%%%%%%%%%%%%%%%%%%%%%%%%%%%%%%%%%%%%%%%%%%%%%%%%%%%%%%%%%%%%%%%
{
	\setbeamertemplate{footline}{}
	\setbeamercolor{background canvas}{bg=ace-blue}
	\setbeamercolor{normal text}{fg=white}
	\usebeamercolor[fg]{normal text}
	\begin{frame}[noframenumbering]
		
		\vspace{2\baselineskip}
		\begin{minipage}{0.7\columnwidth}
			\titlepage
		\end{minipage}
		\begin{minipage}{0.25\columnwidth}
			\begin{center}
				\def\logowidth{0.9\columnwidth}
				\includegraphics[width=\logowidth]{Figures/wpi-logo-white}
				
				\vspace{2\baselineskip}
				\includegraphics[width=\logowidth]{Figures/ace-logo}
			\end{center}
		
		\end{minipage}
		
		\vspace{\baselineskip}
		\begin{center}
		
		\parbox{0.8\columnwidth}{ \linespread{1} \Tiny\it\par \textbf{Fair Use Disclaimer:}
			This document may contain copyrighted material, such as photographs and
			diagrams, the use of which may not always have been specifically authorized
			by the copyright owner. \\ The use of copyrighted material in this document
			is in accordance with the ``fair use doctrine'' as incorporated in Title 17
			USC \S 107 of the United States Copyright Act of 1976.}
		\end{center}
		
	\end{frame}
}



\setbeamertemplate{footline}{
	\leavevmode%
	\hbox{%
		\begin{beamercolorbox}[wd=.2\paperwidth,ht=2.25ex,dp=1ex,center]{author in head/foot}%
			\usebeamerfont{author in head/foot} \insertshortauthor
			\end{beamercolorbox}%
		\begin{beamercolorbox}[wd=.6\paperwidth,ht=2.25ex,dp=1ex,center]{title in head/foot}%
			\usebeamerfont{title in head/foot} \insertshorttitle
		\end{beamercolorbox}%
		\begin{beamercolorbox}[wd=0.2\paperwidth,ht=2.25ex,dp=1ex,right]{footer}%
%			\usebeamerfont{wpi-red}\insertshortdate{}\hspace*{2em}
			\insertframenumber{} / \inserttotalframenumber\hspace*{2ex} 
	\end{beamercolorbox}}%
	\vskip0pt%
}




%%%%%%%%%%%%%%%% NEW FRAME %%%%%%%%%%%%%%%%%%%%%%%%%%%%%%%%%%%%%%%%%%%%%%%%%%%%%%%%%%%%%%%%%%%
\begin{frame}
	\frametitle{Introduction}
	\begin{itemize}[<+->]
		\itemsep\baselineskip
		\item Control and estimation methods usually start with a \\ dynamical system 
		model of the form
		\begin{align*}
			x(k + 1) = f( x(k), u(k), w(k) ),
		\end{align*}
		where%
		\begin{itemize}
			\item $x$ is what we call the \alert{state}
			\item $u$ is what we call the \alert{control}
			\item $w$ is what we call the \alert{process noise}
		\end{itemize}
	
		\item Usual assumptions include:
		\begin{itemize}
			\item $x(k) \in \real[n], ~u(k) \in \real[m],$  where $n,m$ are ``small'' $\sim 10$
			\item $f$ is known, e.g., from physics
			\item $w$ is a white noise process, i.e., $w(k) \sim N(0, Q)$ and $w(k), w(\ell)$
			are uncorrelated
		\end{itemize}

	\end{itemize}
	
\end{frame}
%---------------------------------------------------------------------------------------------



%%%%%%%%%%%%%%%% NEW FRAME %%%%%%%%%%%%%%%%%%%%%%%%%%%%%%%%%%%%%%%%%%%%%%%%%%%%%%%%%%%%%%%%%%%
\begin{frame}
	\frametitle{Nominal Model}
	\begin{itemize}[<+->]
		\itemsep\baselineskip
		\item Control and estimation methods usually start with a \\ dynamical system 
		model of the form
		\begin{align*}
			x(k + 1) = f( x(k), u(k), w(k) ),
		\end{align*}
		where%
		\begin{itemize}
			\item $x$ is what we call the {state}
			\item $u$ is what we call the {control}
			\item $w$ is what we call the {process noise}
		\end{itemize}
		
		\item Let us call this the \alert{nominal model} \\
		(to distinguish from other ``models'' that come up)
		
	\end{itemize}
	
\end{frame}
%---------------------------------------------------------------------------------------------



%%%%%%%%%%%%%%%% NEW FRAME %%%%%%%%%%%%%%%%%%%%%%%%%%%%%%%%%%%%%%%%%%%%%%%%%%%%%%%%%%%%%%%%%%%
\begin{frame}
	\frametitle{Nominal v/s Real}
	\begin{itemize}[<+->]
		\itemsep\baselineskip
		\item The real system evolves \emph{close to,} 
		but not \emph{exactly} according to the nominal model. Examples:
		
		\item $f$ is slightly different, e.g., has some extra terms 
		due to simplifications, approximations, epistemic uncertainties
		
		\item $w$ is not white noise
		
		\item There are extra (hidden) states, e.g., actuator dynamics
		
		
	\end{itemize}
	
\end{frame}
%---------------------------------------------------------------------------------------------






%%%%%%%%%%%%%%%% NEW FRAME %%%%%%%%%%%%%%%%%%%%%%%%%%%%%%%%%%%%%%%%%%%%%%%%%%%%%%%%%%%%%%%%%%%
\begin{frame}
	\frametitle{The Uncertainty Propagation Problem}
	\begin{itemize}[<+->]
		\itemsep\baselineskip
		\item We are usually interested in the conditional distribution
		$p(x(k + 1) ~|~ x(k), u(k))$ \\
		for designing estimators or controllers (esp. RL-based controllers)
		\begin{itemize}
			\item Estimators like EKF / UKF need the mean and variance
			\item Particle filter, RL methods need samples 
		\end{itemize}
		
		\item \alert{Option 1} (most common practice): Find this distribution 
		from the nominal model
		\begin{itemize}
			\item This approach involves approximations when $f$ is nonlinear:
			e.g., linearization or unscented transform
			\item Essentially, we entirely neglect the differences between reality 
			and nominal model and hope that the controller / estimator is robust enough
		\end{itemize}
		
		\item \alert{Option 2} (the adaptive control approach): Append a ``structured
		uncertainty'' term $\Phi$ to $f$ and learn its parameters through open- and 
		closed-loop training
		\begin{itemize}
			\item $\Phi$ often resembles a single-layer NN, e.g., weighted sum of RBFs
			
			\item Sometimes used in estimation as well
					
		\end{itemize}
		
	\end{itemize}
	
\end{frame}
%---------------------------------------------------------------------------------------------





%%%%%%%%%%%%%%%% NEW FRAME %%%%%%%%%%%%%%%%%%%%%%%%%%%%%%%%%%%%%%%%%%%%%%%%%%%%%%%%%%%%%%%%%%%
\begin{frame}
	\frametitle{Data}
	\begin{itemize}[<+->]
		\itemsep \baselineskip
		
		\item We have some \alert{real trajectories}, i.e., time series 
		$\tilde{\xi}(k) = (\tilde{x}(k), \tilde{u}(k))$ recorded during 
		operation of the real system
		\begin{itemize}
			\item These are scarce
		\end{itemize}
		
		\item We have some \alert{nominal trajectories}, time series 
		$\xi(k) = (x(k), u(k))$ resulting from simulation of the nominal model
		\begin{itemize}
			\item These are abundant
		\end{itemize}
		
	\end{itemize}
	
\end{frame}
%---------------------------------------------------------------------------------------------





%%%%%%%%%%%%%%%% NEW FRAME %%%%%%%%%%%%%%%%%%%%%%%%%%%%%%%%%%%%%%%%%%%%%%%%%%%%%%%%%%%%%%%%%%%
\begin{frame}
	\frametitle{Rationale}
	\begin{itemize}[<+->]
		\itemsep \baselineskip
		
		\item The unscented transform is motivated by the idea that it is better to
		approximate the \emph{distribution} propagated through $f,$ than to approximate
		$f$ itself (e.g., by linearization)
		
		\item Which suggests that Options~1 and~2 may be outperformed by ...
		
		\item \alert{Option 3:} Train a conditioned generative model on both
		real and nominal trajectories to produce samples of 
		$x(k + 1)$ for given $x(k), u(k)$
	
		
	\end{itemize}
	
\end{frame}
%---------------------------------------------------------------------------------------------







%%%%%%%%%%%%%%%% NEW FRAME %%%%%%%%%%%%%%%%%%%%%%%%%%%%%%%%%%%%%%%%%%%%%%%%%%%%%%%%%%%%%%%%%%%
\begin{frame}
	\frametitle{Case Study 1}
	\begin{itemize}[<+->]
		\itemsep \baselineskip
		
		\item Nominal model is linear:
		\begin{align*}
			x(k + 1) &= A x(k) + B u(k)
		\end{align*}
		
		\item Suppose $x(k) \in \real[2], ~u(k) \in \real$ and $A$ is Hurwitz: 
		$Re$(both eigenvalues) $< 0$
		
		\item Real system does not have process noise, but has unmodeled terms:
		\begin{align*}
			x(k + 1) &= A x(k) + B u(k) + \mathalert{\tilde{f}(x(k))}
		\end{align*}
		
		
	\end{itemize}
	
\end{frame}
%---------------------------------------------------------------------------------------------










%%%%%%%%%%%%%%%% NEW FRAME %%%%%%%%%%%%%%%%%%%%%%%%%%%%%%%%%%%%%%%%%%%%%%%%%%%%%%%%%%%%%%%%%%%
\begin{frame}
	\frametitle{Questions for Randy}
	
	\begin{minipage}{0.7\columnwidth}
	\begin{itemize}[<+->]
		\itemsep \baselineskip
		
		\item Are there generative models in other domains (e.g., text / image)
		that address this type of a problem?
		
		\item Perhaps a ``variational propagator''?
		\begin{align*}
			\xi=(x,u) &\rightarrow p(z | \xi) &
			z \sim p(z | \xi) &\rightarrow D(z)
		\end{align*}
		\begin{itemize}
			\item Loss$ = \lambda_1 \| D(z) - \mathalert{x(k+1)} \| + 
			\lambda_2 KL(p~|| \mathcal{N}(0,I) ) + \lambda_3 \underbrace{\|x - f(x,u)\|}_{\text{or } \left| \|x 
			- f(x,u) \| - \varepsilon \right|}$
		\end{itemize}
		
		\item How can we ``condition'' this on $x(k), u(k)$?
	\end{itemize}
	\end{minipage}
	\begin{minipage}{0.25\columnwidth}
		\centering
		\vspace*{2\baselineskip}
		\includegraphics[width=0.8\columnwidth]{Figures/kneelingprayer}
	\end{minipage}


\end{frame}
%---------------------------------------------------------------------------------------------

%
%
%
%%%%%%%%%%%%%%%%% NEW FRAME %%%%%%%%%%%%%%%%%%%%%%%%%%%%%%%%%%%%%%%%%%%%%%%%%%%%%%%%%%%%%%%%%%%
%\begin{frame}
%	\frametitle{Practices to Follow and Avoid (1/2)}
%	
%	\begin{minipage}{0.6\columnwidth}
%		\begin{itemize}[<+->]
%			\itemsep \baselineskip
%			
%			\item Bulleted lists should appear one bullet at a time.
%			\begin{itemize}
%				\item Use \texttt{handout} option in \texttt{$\backslash$documentclass}
%				to save time while developing the presentation; remove that option to create the final PDF.
%				\item Each bullet should be exactly one sentence.
%				\item Use multiple bullets and sub-bullets as needed.
%			\end{itemize}
%			
%			\item Use text sparingly.
%			\begin{itemize}
%				\item Instead of text, consider using relevant images.
%				\item Not everything you want to say needs to appear on the slide.
%				\item Your oral presentation should be more than just reading the slides aloud.
%			\end{itemize}
%			
%			
%		\end{itemize}
%	\end{minipage}
%	\hspace{0.01\columnwidth}
%	\begin{minipage}{0.34\columnwidth}
%		\begin{center}
%			\includegraphics[width=\columnwidth]{Figures/more-less}
%			\figcaption{
%			Source: https://nextadagency.com/less-is-more-when-it-comes-to-your-website-content/
%			}
%		\end{center}
%		
%		
%		
%	\end{minipage}
%	
%\end{frame}
%%---------------------------------------------------------------------------------------------
%
%
%
%%%%%%%%%%%%%%%%% NEW FRAME %%%%%%%%%%%%%%%%%%%%%%%%%%%%%%%%%%%%%%%%%%%%%%%%%%%%%%%%%%%%%%%%%%%
%%\begin{frame}
%%	\frametitle{Title}
%%	\vspace{-8\baselineskip}
%%	How to pronounce my last name: ``cow-luh-gee''
%%	\vspace{\baselineskip}
%%	
%%	A brief bio:
%%	
%%\end{frame}
%%---------------------------------------------------------------------------------------------
%
%
%
%
%%%%%%%%%%%%%%%%% NEW FRAME %%%%%%%%%%%%%%%%%%%%%%%%%%%%%%%%%%%%%%%%%%%%%%%%%%%%%%%%%%%%%%%%%%%
%\begin{frame}
%	\frametitle{Practices to Follow and Avoid (2/2)}
%	
%	\begin{itemize}[<+->]
%		\itemsep \baselineskip
%		\item Beamer is designed to ensure visual consistency.
%		\begin{itemize}
%			\item Do not manually change font sizes, colors, etc., 
%			unless \alert{absolutely} needed.
%			\item If you feel the need to reduce the font size or spacing, 
%			you are probably trying to put too much text in one slide.
%		\end{itemize}
%	
%		\item Explain all notation at first use for all math. equations / expressions, e.g.,
%		$$ c(\vec{p},t) = \Phi^\intercal(\vec{p}) \Theta(t). $$
%		\begin{itemize}
%			\item $c$ is a threat field defined over space $\vec{p}$ 
%			and time $t,$
%			\item $\Phi$ is a basis function vector and $\Theta$ is
%			a parameter vector.
%		\end{itemize}
%	\end{itemize}
%	
%\end{frame}
%%---------------------------------------------------------------------------------------------
%
%
%
%
%%%%%%%%%%%%%%%%% NEW FRAME %%%%%%%%%%%%%%%%%%%%%%%%%%%%%%%%%%%%%%%%%%%%%%%%%%%%%%%%%%%%%%%%%%%
%\begin{frame}
%	\frametitle{Resources}
%	
%	\begin{itemize}[<+->]
%		\itemsep \baselineskip
%		\item \url{https://www.overleaf.com/learn/latex/Beamer}
%		
%		\item 
%		
%\url{https://www.wiley.com/en-us/network/publishing/research-publishing/writing-and-conducting-research/6-tips-for-giving-a-fabulous-academic-presentation}
%		
%		\item The source code for this presentation has some examples of typical slide layouts.
%	\end{itemize}
%	
%\end{frame}
%%---------------------------------------------------------------------------------------------
%
%
%%%%%%%%%%%%%%%%% NEW FRAME %%%%%%%%%%%%%%%%%%%%%%%%%%%%%%%%%%%%%%%%%%%%%%%%%%%%%%%%%%%%%%%%%%%
%\begin{frame}
%	\frametitle{Example of Bulleted List}
%	
%	\begin{itemize}[<+->]
%		\itemsep \baselineskip
%		\item Each bullet appears one at a time due to the \texttt{[<+->]} option
%		in the \texttt{itemize} environment.
%		
%		\item Bullet 2 $\ldots$
%		
%		\item Bullet 3 $\ldots$
%		\begin{itemize}
%			\item Sub-bullet 1 $\ldots$
%			\item Learn about \texttt{overlay} and \texttt{pause} for more
%			control  over \\the sequence of appearance of various items on the slide.
%		\end{itemize}
%	\end{itemize}
%	
%\end{frame}
%%---------------------------------------------------------------------------------------------
%
%
%%%%%%%%%%%%%%%%% NEW FRAME %%%%%%%%%%%%%%%%%%%%%%%%%%%%%%%%%%%%%%%%%%%%%%%%%%%%%%%%%%%%%%%%%%%
%\begin{frame}
%	\frametitle{Example of a Slide with a Single Image}
%	
%	\begin{center}
%		\includegraphics[width=0.5\columnwidth]{Figures/sample-threat}
%		\figcaption{Insert a concise and relevant caption. 
%			Include source if it is not your original image.}
%	\end{center}
%	
%\end{frame}
%%---------------------------------------------------------------------------------------------
%
%
%
%%%%%%%%%%%%%%%%% NEW FRAME %%%%%%%%%%%%%%%%%%%%%%%%%%%%%%%%%%%%%%%%%%%%%%%%%%%%%%%%%%%%%%%%%%%
%\begin{frame}
%	\frametitle{Example of a Slide with Juxtaposed Images}
%	
%	
%		\begin{figure}
%			\centering
%			\subfigure{\includegraphics[width=0.25\columnwidth]{Figures/sample-threat}}
%			\hspace{0.05\columnwidth}
%			\subfigure{\includegraphics[width=0.25\columnwidth]{Figures/sample-threat}}
%		\end{figure}		
%		\figcaption{Use \texttt{subfigure} or to put multiple images. Next slide
%		uses the \texttt{picture} environment for more complex image/text layouts.}
%	
%\end{frame}
%%---------------------------------------------------------------------------------------------
%
%
%
%%%%%%%%%%%%%%%%% NEW FRAME %%%%%%%%%%%%%%%%%%%%%%%%%%%%%%%%%%%%%%%%%%%%%%%%%%%%%%%%%%%%%%%%%%%
%\begin{frame}
%	\frametitle{Example of a Non-Standard Layout with Images and Text}
%	
%		\begin{picture}(0,0)(-10,-30)
%			\put(0,0){\includegraphics[width=0.3\columnwidth]{Figures/logo-gatech}}
%			\put(150,7){\parbox{0.75\columnwidth}{\footnotesize \begin{itemize}
%									\listformat \item Ph.D., 2011. \item Research on autonomous aircraft.
%						\end{itemize}}}
%			
%			\put(0,-50){\includegraphics[width=0.3\columnwidth]{Figures/logo-mit}}
%			\put(150,-40){\parbox{0.75\columnwidth}{\footnotesize \begin{itemize}
%									\listformat \item Postdoc, 2011 -- 12. \item Research on autonomous cars.
%						\end{itemize}}}
%			
%			\put(0,-100){\includegraphics[width=0.3\columnwidth]{Figures/logo-aurora}}
%			\put(150,-90){\parbox{0.75\columnwidth}{\footnotesize \begin{itemize}
%									\listformat \item 2012 -- 13. \item Research on autonomous aircraft.
%						\end{itemize}}}
%		\end{picture}
%
%	
%\end{frame}
%%---------------------------------------------------------------------------------------------
%
%
%%%%%%%%%%%%%%%%% NEW FRAME %%%%%%%%%%%%%%%%%%%%%%%%%%%%%%%%%%%%%%%%%%%%%%%%%%%%%%%%%%%%%%%%%%%
%\begin{frame}
%	\frametitle{Example of a Slide with a Main Mathematical Result}
%	
%	\begin{center}
%		\begin{minipage}{0.8\columnwidth}
%			
%				\begin{theorem}[Convergence]
%					The CSCP algorithm converges in a finite number of iterations and
%					$$ \mathbb{P}[J(u^*) \leq \varepsilon] \geq 0.99. $$
%				\end{theorem}
%				\pause
%			
%				Save this block environment for important results. Do not overuse.
%		\end{minipage}
%	\end{center}
%	
%\end{frame}
%%---------------------------------------------------------------------------------------------



\end{document}

