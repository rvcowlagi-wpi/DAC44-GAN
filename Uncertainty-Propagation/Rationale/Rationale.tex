\documentclass[aspectratio=169,handout]{beamer} 	% <== Use this during development to save time
%\documentclass[aspectratio=169]{beamer}		% <== Use this for final compilation (comment out previous)


\usepackage{color}
\usepackage{subfigure}
\usepackage{soul}
\usepackage{upgreek}
\usepackage{tikz}
\usepackage{ctable}


\mode<presentation>

%========== Presentation Theme ===========
\definecolor{wpi-red}{RGB}{175, 43, 55}
\definecolor{ace-blue}{RGB}{31,52,77}
\definecolor{ace-yellow}{RGB}{254,250,201}
\definecolor{my-gray}{gray}{0.9}

\usetheme{boxes}
\usecolortheme{orchid}

\setbeamercolor{title}{fg=ace-yellow}
\setbeamercolor{frametitle}{fg=white,bg=ace-blue}
\setbeamercolor{alerted text}{fg=wpi-red}
\setbeamercolor{normal text}{fg=black}
\setbeamercolor{item}{fg=black}
\setbeamercolor{footer}{fg=white,bg=ace-blue}
\setbeamercolor{author in head/foot}{fg=white,bg=ace-blue}
\setbeamercolor{title in head/foot}{fg=ace-blue,bg=my-gray}

\setbeamerfont{frametitle}{size=\Large}
\setbeamerfont{alerted text}{shape=\itshape}

\usepackage{charter}
\usepackage[T1]{fontenc}
\usefonttheme{serif}
%\usefonttheme[onlymath]{serif}


\setbeamertemplate{frametitle}[default][center]
\setbeamertemplate{navigation symbols}{}
\setbeamertemplate{caption}{\centering\tiny\insertcaption\par}


\setlength{\fboxrule}{1.5pt}
\setlength\abovecaptionskip{-0.3\baselineskip}
\setlength\belowcaptionskip{0.25\baselineskip}

\addtobeamertemplate{frametitle}{\vspace*{0\baselineskip}}{\vspace*{-0.06\textwidth}
	\hspace*{0.9345\textwidth}\includegraphics[width=0.1\textwidth]{Figures/ace-logo}\vspace*{-0.06\textwidth}}


\setbeamercolor{block title example}{fg=white,	bg=ace-blue}
\setbeamercolor{block title}{fg=white,	bg=wpi-red}
\setbeamercolor{block body example}{parent=normal text,use=block title,bg=block title.bg!10!bg}
%\setbeamercolor{date in head/foot}{fg=wpi-red}


\setbeamersize{text margin left=3ex,text margin right=3ex}

\newcommand{\mathalert}[1]{{\color{wpi-red}#1}}
\newcommand{\figcaption}[1]{\parbox{\columnwidth}{\centering \tiny\linespread{0.5} #1}{}}

\setlength{\fboxrule}{1pt}
\setlength{\fboxsep}{0pt}



\setbeamertemplate{footline}{
	\leavevmode%
	\hbox{%
		\begin{beamercolorbox}[wd=.2\paperwidth,ht=2.25ex,dp=1ex,left]{author in head/foot}%
			\usebeamerfont{author in head/foot} \hspace*{3ex}\insertshortauthor
		\end{beamercolorbox}%
		\begin{beamercolorbox}[wd=.6\paperwidth,ht=2.25ex,dp=1ex,center]{title in head/foot}%
			\usebeamerfont{title in head/foot} \insertshorttitle
		\end{beamercolorbox}%
		\begin{beamercolorbox}[wd=0.2\paperwidth,ht=2.25ex,dp=1ex,right]{footer}%
			%			\usebeamerfont{wpi-red}\insertshortdate{}\hspace*{2em}
			\insertframenumber{} / \inserttotalframenumber\hspace*{3ex} 
	\end{beamercolorbox}}%
	\vskip0pt%
}
\input{Macros}




\begin{document}



\title[] 					% <= Short title here for footer on each slide
{\LARGE \bf Thoughts about Generative Models in Control Systems \\   	% <= Presentation title
	{\small \it }} 	% <= Optional subtitle
\author[Prof. R. V. Cowlagi]{{Prof. Raghvendra V. Cowlagi}\inst{}} 		% <== Author information
\institute[]{\inst{}Aerospace Engineering Department,\\
	Worcester Polytechnic Institute, Worcester, MA.\\
	\textbf{rvcowlagi@wpi.edu} \qquad \qquad		% <== Contact information
	\textbf{wpi.edu/$\sim$rvcowlagi} \qquad  		% <== Contact information
	\qquad \textbf{HL 247}}


%%%%%%%%%%%%%%%% TITLE PAGE %%%%%%%%%%%%%%%%%%%%%%%%%%%%%%%%%%%%%%%%%%%%%%%%%%%%%%%%%%%%%%%%%%%
{
	\setbeamertemplate{footline}{}
	\setbeamercolor{background canvas}{bg=ace-blue}
	\setbeamercolor{normal text}{fg=white}
	\usebeamercolor[fg]{normal text}
	\begin{frame}[noframenumbering]
		
		\vspace{2\baselineskip}
		\begin{minipage}{0.7\columnwidth}
			\titlepage
		\end{minipage}
		\begin{minipage}{0.25\columnwidth}
			\begin{center}
				\def\logowidth{0.9\columnwidth}
				\includegraphics[width=\logowidth]{Figures/wpi-logo-white}
				
				\vspace{2\baselineskip}
				\includegraphics[width=\logowidth]{Figures/ace-logo}
			\end{center}
		
		\end{minipage}
		
		\vspace{\baselineskip}
		\begin{center}
		
		\parbox{0.8\columnwidth}{ \linespread{1} \Tiny\it\par \textbf{Fair Use Disclaimer:}
			This document may contain copyrighted material, such as photographs and
			diagrams, the use of which may not always have been specifically authorized
			by the copyright owner. \\ The use of copyrighted material in this document
			is in accordance with the ``fair use doctrine'' as incorporated in Title 17
			USC \S 107 of the United States Copyright Act of 1976.}
		\end{center}
		
	\end{frame}
}



\setbeamertemplate{footline}{
	\leavevmode%
	\hbox{%
		\begin{beamercolorbox}[wd=.2\paperwidth,ht=2.25ex,dp=1ex,center]{author in head/foot}%
			\usebeamerfont{author in head/foot} \insertshortauthor
			\end{beamercolorbox}%
		\begin{beamercolorbox}[wd=.6\paperwidth,ht=2.25ex,dp=1ex,center]{title in head/foot}%
			\usebeamerfont{title in head/foot} \insertshorttitle
		\end{beamercolorbox}%
		\begin{beamercolorbox}[wd=0.2\paperwidth,ht=2.25ex,dp=1ex,right]{footer}%
%			\usebeamerfont{wpi-red}\insertshortdate{}\hspace*{2em}
			\insertframenumber{} / \inserttotalframenumber\hspace*{2ex} 
	\end{beamercolorbox}}%
	\vskip0pt%
}




%%%%%%%%%%%%%%%% NEW FRAME %%%%%%%%%%%%%%%%%%%%%%%%%%%%%%%%%%%%%%%%%%%%%%%%%%%%%%%%%%%%%%%%%%%
\begin{frame}
	\frametitle{}
	\begin{itemize}[<+->]
		\itemsep\baselineskip
		\item Control and estimation methods usually start with a \\ dynamical system 
		model of the form
		\begin{align*}
			x(k + 1) = f( x(k), u(k), w(k) ),
		\end{align*}
		where%
		\begin{itemize}
			\item $x$ is what we call the \alert{state}
			\item $u$ is what we call the \alert{control}
			\item $w$ is what we call the \alert{process noise}
		\end{itemize}
	
		\item Usual assumptions include:
		\begin{itemize}
			\item $x(k) \in \real[n], ~u(k) \in \real[m],$  where $n,m$ are ``small'' $\sim 10$
			\item $f$ is known, e.g., from physics
			\item $w$ is a white noise process, i.e., $w(k) \sim N(0, Q)$ and $w(k), w(\ell)$
			are uncorrelated
		\end{itemize}

	\end{itemize}
	
\end{frame}
%---------------------------------------------------------------------------------------------



%%%%%%%%%%%%%%%% NEW FRAME %%%%%%%%%%%%%%%%%%%%%%%%%%%%%%%%%%%%%%%%%%%%%%%%%%%%%%%%%%%%%%%%%%%
\begin{frame}
	\frametitle{}
	\begin{itemize}[<+->]
		\itemsep\baselineskip
		\item Control and estimation methods usually start with a \\ dynamical system 
		model of the form
		\begin{align*}
			x(k + 1) = f( x(k), u(k), w(k) ),
		\end{align*}
		where%
		\begin{itemize}
			\item $x$ is what we call the {state}
			\item $u$ is what we call the {control}
			\item $w$ is what we call the {process noise}
		\end{itemize}
		
		\item Let us call this the \alert{believed model} \\
		(to distinguish from other ``models'' that come up)
		
	\end{itemize}
	
\end{frame}
%---------------------------------------------------------------------------------------------



%%%%%%%%%%%%%%%% NEW FRAME %%%%%%%%%%%%%%%%%%%%%%%%%%%%%%%%%%%%%%%%%%%%%%%%%%%%%%%%%%%%%%%%%%%
\begin{frame}
	\frametitle{}
	\begin{itemize}[<+->]
		\itemsep\baselineskip
		\item The real system evolves \emph{close to,} 
		but not \emph{exactly} according to the believed model. Examples:
		
		\item $f$ is slightly different, e.g., has some extra terms 
		due to simplified physics
		
		\item $w$ is not white noise
		
		\item There are extra (hidden) states, e.g., actuator dynamics
		
		
	\end{itemize}
	
\end{frame}
%---------------------------------------------------------------------------------------------





%%%%%%%%%%%%%%%% NEW FRAME %%%%%%%%%%%%%%%%%%%%%%%%%%%%%%%%%%%%%%%%%%%%%%%%%%%%%%%%%%%%%%%%%%%
\begin{frame}
	\frametitle{}
	\begin{itemize}[<+->]
		\itemsep\baselineskip
		\item We are usually interested in the conditional distribution
		$p(x(k + 1) ~|~ x(k), u(k))$ \\
		for designing estimators or controllers (esp. RL-based controllers)
		
		
	\end{itemize}
	
\end{frame}
%---------------------------------------------------------------------------------------------



%
%
%%%%%%%%%%%%%%%%% NEW FRAME %%%%%%%%%%%%%%%%%%%%%%%%%%%%%%%%%%%%%%%%%%%%%%%%%%%%%%%%%%%%%%%%%%%
%\begin{frame}
%	\frametitle{Example of an Overview Slide}
%	\begin{itemize}[<+->]
%		\itemsep \baselineskip
%		
%		\item List \alert{no more than 4} main sections, examples below.
%		\begin{itemize}
%			\item No periods (full stops) after section title bullet list.
%			\item Highlight the section to follow immediately after this slide.
%			\item Previous sections should be in gray color. See examples below.
%			\item ``Introduction'' is not a section; this Overview is part of
%			an ``Introduction''.
%		\end{itemize}
%		
%		\item \textcolor{my-gray}{Problem Formulation}
%		
%		\item \textbf{Method}
%		
%		\item Results \& Discussion
%	\end{itemize}
%	
%\end{frame}
%%---------------------------------------------------------------------------------------------
%
%
%
%
%
%%%%%%%%%%%%%%%%% NEW FRAME %%%%%%%%%%%%%%%%%%%%%%%%%%%%%%%%%%%%%%%%%%%%%%%%%%%%%%%%%%%%%%%%%%%
%\begin{frame}
%	\frametitle{Practices to Follow and Avoid (1/2)}
%	
%	\begin{minipage}{0.6\columnwidth}
%		\begin{itemize}[<+->]
%			\itemsep \baselineskip
%			
%			\item Bulleted lists should appear one bullet at a time.
%			\begin{itemize}
%				\item Use \texttt{handout} option in \texttt{$\backslash$documentclass}
%				to save time while developing the presentation; remove that option to create the final PDF.
%				\item Each bullet should be exactly one sentence.
%				\item Use multiple bullets and sub-bullets as needed.
%			\end{itemize}
%			
%			\item Use text sparingly.
%			\begin{itemize}
%				\item Instead of text, consider using relevant images.
%				\item Not everything you want to say needs to appear on the slide.
%				\item Your oral presentation should be more than just reading the slides aloud.
%			\end{itemize}
%			
%			
%		\end{itemize}
%	\end{minipage}
%	\hspace{0.01\columnwidth}
%	\begin{minipage}{0.34\columnwidth}
%		\begin{center}
%			\includegraphics[width=\columnwidth]{Figures/more-less}
%			\figcaption{
%			Source: https://nextadagency.com/less-is-more-when-it-comes-to-your-website-content/
%			}
%		\end{center}
%		
%		
%		
%	\end{minipage}
%	
%\end{frame}
%%---------------------------------------------------------------------------------------------
%
%
%
%%%%%%%%%%%%%%%%% NEW FRAME %%%%%%%%%%%%%%%%%%%%%%%%%%%%%%%%%%%%%%%%%%%%%%%%%%%%%%%%%%%%%%%%%%%
%%\begin{frame}
%%	\frametitle{Title}
%%	\vspace{-8\baselineskip}
%%	How to pronounce my last name: ``cow-luh-gee''
%%	\vspace{\baselineskip}
%%	
%%	A brief bio:
%%	
%%\end{frame}
%%---------------------------------------------------------------------------------------------
%
%
%
%
%%%%%%%%%%%%%%%%% NEW FRAME %%%%%%%%%%%%%%%%%%%%%%%%%%%%%%%%%%%%%%%%%%%%%%%%%%%%%%%%%%%%%%%%%%%
%\begin{frame}
%	\frametitle{Practices to Follow and Avoid (2/2)}
%	
%	\begin{itemize}[<+->]
%		\itemsep \baselineskip
%		\item Beamer is designed to ensure visual consistency.
%		\begin{itemize}
%			\item Do not manually change font sizes, colors, etc., 
%			unless \alert{absolutely} needed.
%			\item If you feel the need to reduce the font size or spacing, 
%			you are probably trying to put too much text in one slide.
%		\end{itemize}
%	
%		\item Explain all notation at first use for all math. equations / expressions, e.g.,
%		$$ c(\vec{p},t) = \Phi^\intercal(\vec{p}) \Theta(t). $$
%		\begin{itemize}
%			\item $c$ is a threat field defined over space $\vec{p}$ 
%			and time $t,$
%			\item $\Phi$ is a basis function vector and $\Theta$ is
%			a parameter vector.
%		\end{itemize}
%	\end{itemize}
%	
%\end{frame}
%%---------------------------------------------------------------------------------------------
%
%
%
%
%%%%%%%%%%%%%%%%% NEW FRAME %%%%%%%%%%%%%%%%%%%%%%%%%%%%%%%%%%%%%%%%%%%%%%%%%%%%%%%%%%%%%%%%%%%
%\begin{frame}
%	\frametitle{Resources}
%	
%	\begin{itemize}[<+->]
%		\itemsep \baselineskip
%		\item \url{https://www.overleaf.com/learn/latex/Beamer}
%		
%		\item 
%		
%\url{https://www.wiley.com/en-us/network/publishing/research-publishing/writing-and-conducting-research/6-tips-for-giving-a-fabulous-academic-presentation}
%		
%		\item The source code for this presentation has some examples of typical slide layouts.
%	\end{itemize}
%	
%\end{frame}
%%---------------------------------------------------------------------------------------------
%
%
%%%%%%%%%%%%%%%%% NEW FRAME %%%%%%%%%%%%%%%%%%%%%%%%%%%%%%%%%%%%%%%%%%%%%%%%%%%%%%%%%%%%%%%%%%%
%\begin{frame}
%	\frametitle{Example of Bulleted List}
%	
%	\begin{itemize}[<+->]
%		\itemsep \baselineskip
%		\item Each bullet appears one at a time due to the \texttt{[<+->]} option
%		in the \texttt{itemize} environment.
%		
%		\item Bullet 2 $\ldots$
%		
%		\item Bullet 3 $\ldots$
%		\begin{itemize}
%			\item Sub-bullet 1 $\ldots$
%			\item Learn about \texttt{overlay} and \texttt{pause} for more
%			control  over \\the sequence of appearance of various items on the slide.
%		\end{itemize}
%	\end{itemize}
%	
%\end{frame}
%%---------------------------------------------------------------------------------------------
%
%
%%%%%%%%%%%%%%%%% NEW FRAME %%%%%%%%%%%%%%%%%%%%%%%%%%%%%%%%%%%%%%%%%%%%%%%%%%%%%%%%%%%%%%%%%%%
%\begin{frame}
%	\frametitle{Example of a Slide with a Single Image}
%	
%	\begin{center}
%		\includegraphics[width=0.5\columnwidth]{Figures/sample-threat}
%		\figcaption{Insert a concise and relevant caption. 
%			Include source if it is not your original image.}
%	\end{center}
%	
%\end{frame}
%%---------------------------------------------------------------------------------------------
%
%
%
%%%%%%%%%%%%%%%%% NEW FRAME %%%%%%%%%%%%%%%%%%%%%%%%%%%%%%%%%%%%%%%%%%%%%%%%%%%%%%%%%%%%%%%%%%%
%\begin{frame}
%	\frametitle{Example of a Slide with Juxtaposed Images}
%	
%	
%		\begin{figure}
%			\centering
%			\subfigure{\includegraphics[width=0.25\columnwidth]{Figures/sample-threat}}
%			\hspace{0.05\columnwidth}
%			\subfigure{\includegraphics[width=0.25\columnwidth]{Figures/sample-threat}}
%		\end{figure}		
%		\figcaption{Use \texttt{subfigure} or to put multiple images. Next slide
%		uses the \texttt{picture} environment for more complex image/text layouts.}
%	
%\end{frame}
%%---------------------------------------------------------------------------------------------
%
%
%
%%%%%%%%%%%%%%%%% NEW FRAME %%%%%%%%%%%%%%%%%%%%%%%%%%%%%%%%%%%%%%%%%%%%%%%%%%%%%%%%%%%%%%%%%%%
%\begin{frame}
%	\frametitle{Example of a Non-Standard Layout with Images and Text}
%	
%		\begin{picture}(0,0)(-10,-30)
%			\put(0,0){\includegraphics[width=0.3\columnwidth]{Figures/logo-gatech}}
%			\put(150,7){\parbox{0.75\columnwidth}{\footnotesize \begin{itemize}
%									\listformat \item Ph.D., 2011. \item Research on autonomous aircraft.
%						\end{itemize}}}
%			
%			\put(0,-50){\includegraphics[width=0.3\columnwidth]{Figures/logo-mit}}
%			\put(150,-40){\parbox{0.75\columnwidth}{\footnotesize \begin{itemize}
%									\listformat \item Postdoc, 2011 -- 12. \item Research on autonomous cars.
%						\end{itemize}}}
%			
%			\put(0,-100){\includegraphics[width=0.3\columnwidth]{Figures/logo-aurora}}
%			\put(150,-90){\parbox{0.75\columnwidth}{\footnotesize \begin{itemize}
%									\listformat \item 2012 -- 13. \item Research on autonomous aircraft.
%						\end{itemize}}}
%		\end{picture}
%
%	
%\end{frame}
%%---------------------------------------------------------------------------------------------
%
%
%%%%%%%%%%%%%%%%% NEW FRAME %%%%%%%%%%%%%%%%%%%%%%%%%%%%%%%%%%%%%%%%%%%%%%%%%%%%%%%%%%%%%%%%%%%
%\begin{frame}
%	\frametitle{Example of a Slide with a Main Mathematical Result}
%	
%	\begin{center}
%		\begin{minipage}{0.8\columnwidth}
%			
%				\begin{theorem}[Convergence]
%					The CSCP algorithm converges in a finite number of iterations and
%					$$ \mathbb{P}[J(u^*) \leq \varepsilon] \geq 0.99. $$
%				\end{theorem}
%				\pause
%			
%				Save this block environment for important results. Do not overuse.
%		\end{minipage}
%	\end{center}
%	
%\end{frame}
%%---------------------------------------------------------------------------------------------



\end{document}

